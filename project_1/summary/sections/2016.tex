% \section{2016 - Varun Lingabathini}
    \section{Semantic Technologies for Data Analysis in Health Care}

    % \url{http://www.cs.ox.ac.uk/people/boris.motik/pubs/pnmhhkr16semantic-health-care.pdf}
    % \bigskip 

    The HMOs in US need to demonstrate satisfactory performance W.r.t NCQA measures if they wish to participate in government funded healthcare schemes. The quality measures proposed are regularly revised and updated and this makes computation of relevant quality measures very complex using existing systems like SQL and SAS tools.
    In order to integrate relevant data from different heterogeneous sources, RDF is used along with declarative rules and adaptable schemas. The first step is to create a data model to transform the relevant patient data to a human readable format and the second step is to generate declarative rules and execute these rules using SPARQL queries.
    The above approach is implemented in one of the health care company in United states known as Kaiser permanente in Georgia region. The data is translated according to the above approach and the results were compared with the existing systems.The results were extremely encouraging as only 174 rules were needed when compared to 3000 lines of complex SQL code.
 

    \section{Extracting Semantic Information for e-Commerce}

    % \url{http://www-kasm.nii.ac.jp/iswc2016/papers/paper_A21_.pdf }
    % \bigskip 
    
    E-commerce websites in general uses a large legacy taxonomy of classes to organize the items and provide the relevant searches. In order to improve the user search experience, there is a need to extend the taxonomy and this takes a massive amount of time and human effort. The author tries to provide a solution to automate the process which would eventually increase the profitability of the business.
    The solution is to extend the taxonomy for the classes which are usually difficult to explore and this is achieved by aggregating small set of properties which are most popular among the users with the automatically selected taxonomy subtree and the outcomes are new RDF triples.
    Comparing the results of the automated approach to the manual work, the results were not consistent enough but using this automated approach along with manual process will help the accuracy get to above 80\%.
